\documentclass[11pt]{article}

\usepackage{enumitem}
\usepackage{amsmath}
\usepackage{amssymb}
\usepackage{bm}
\usepackage{listings}
\usepackage{tikz}
\usepackage[T1]{fontenc}
\usepackage{courier}
\usepackage{circuitikz}
\usetikzlibrary{calc}

\lstset{xleftmargin=-1cm,
	xrightmargin=\parindent,
	numbersep=5pt} 

\title{CS152B Lab 1}

\begin{document}
	
\title{\vspace{-0.5in} Com Sci 152B Digital Design \\
	Lab 1: ALU \& Register File Implementation }
\date{}
\maketitle
\vspace{-0.75in}
\begin{center}
\begin{tabular}{cc}
	Michael Hale & 004-620-459 \\ 
	Matthew Nuesca & 904-440-067 \\ 
	Shilin Patel & 904-569-866 \\ 
	Bingxin Zhu & 704-845-969
\end{tabular}
\end{center}

\section*{Overview}

In this lab we implemented two key components found in modern processors: the arithmetic logic unit (ALU) and the register file. The purpose of the ALU is to perform a variety of operations on arbitrary 16-bit integers and write the result to an output port. On the other hand, the register file performs no computation and simply reads and writes to a series of internal 16-bit registers. 

\tikzset{
	multiplexer/.pic = {
		
		% frame
		\draw[pic actions] (0, 0) coordinate (-NW) -- ++(300 : 1.5)
		coordinate (-SW) -- ++(0 : 1.5) coordinate (-SE) -- ++(60 : 1.5)
		coordinate (-NE) -- cycle;
		
		% output
		\coordinate (-output) at ($(-SW)!0.5!(-SE)$);
		
		% select
		\foreach \x/\lbl in {0.33/A,0.66/B} {
			\coordinate (-\lbl) at ($(-SE)!\x!(-NE)$);
			
		}
		
		% input
		\pgfmathsetmacro{\ymin}{0.75}
		\pgfmathsetmacro{\ymax}{2.25}
		\foreach \i in {0,1} {
			\foreach \j in {0,1} {
				\pgfmathsetmacro{\y}{\ymax - (\ymax - \ymin)*(2*\i + \j) / 3.};
				\coordinate (-\i\j) at (\y,0);
				\node[yshift = -8pt] at (-\i\j) {$\i\j$};
			}
		}
	}
}


\section*{Shifter}
\begin{tikzpicture}

\draw (0,0) pic(multi3) [draw, fill = white] {multiplexer};
\draw (3,0) pic(multi2) [draw, fill = white] {multiplexer};
\draw (6,0) pic(multi1) [draw, fill = white] {multiplexer};
\draw (9,0) pic(multi0) [draw, fill = white] {multiplexer};

% select
\coordinate (T) at ([xshift = 5pt]multi0-NE);
\draw (multi0-A) -- (multi0-A -| T) node[right]{$B_0$};
\draw (multi0-B) -- (multi0-B -| T) node[right]{$B_1$};
%\foreach \s in {A,B} 
%\draw (multi1-\s) -- (multi1-\s -|  T) node[right]{$\s$};
\coordinate (A0) at (10.5,5);
\coordinate (A1) at (7.5,5);
\coordinate (A2) at (4.5,5);
\coordinate (A3) at (1.5,5);

\coordinate (N0) at (10.5,4);
\coordinate (N1) at (7.5,4);
\coordinate (N2) at (4.5,4);
\coordinate (N3) at (1.5,4);
\foreach \a in {0,1,2,3}
	\draw (N\a) -- (A\a) node[above]{$A_\a$};
\foreach \m in {0,1,2,3} {
	\foreach \i in {00,01,10,11} {
		\draw (multi\m-\i) -- ++(90:0.5) coordinate (multi\m-\i-raised);
	}	
}
\draw (multi0-00-raised) -- (N0);
\draw (multi1-00-raised) -- (N1);
\draw (multi0-01-raised) -- (N1);
\draw (multi2-00-raised) -- (N2);
\draw (multi1-01-raised) -- (N2);
\draw (multi0-10-raised) -- (N2);
\draw (multi3-00-raised) -- (N3);
\draw (multi2-01-raised) -- (N3);
\draw (multi1-10-raised) -- (N3);
\draw (multi0-11-raised) -- (N3);

\draw (multi3-11-raised) node[above]{$0$};
\draw (multi3-10-raised) node[above]{$0$};
\draw (multi3-01-raised) node[above]{$0$};
\draw (multi2-11-raised) node[above]{$0$};
\draw (multi2-10-raised) node[above]{$0$};
\draw (multi1-11-raised) node[above]{$0$};

% output
\draw (multi3-output) -- ++(270:0.5) node[below]{$S_3$};
\draw (multi2-output) -- ++(270:0.5) node[below]{$S_2$};
\draw (multi1-output) -- ++(270:0.5) node[below]{$S_1$};
\draw (multi0-output) -- ++(270:0.5) node[below]{$S_0$};

\end{tikzpicture}

\section*{Arithmetic Right Shifter}
\begin{tikzpicture}

\draw (0,0) pic(multi3) [draw, fill = white] {multiplexer};
\draw (3,0) pic(multi2) [draw, fill = white] {multiplexer};
\draw (6,0) pic(multi1) [draw, fill = white] {multiplexer};
\draw (9,0) pic(multi0) [draw, fill = white] {multiplexer};

% select
\coordinate (T) at ([xshift = 5pt]multi0-NE);
\draw (multi0-A) -- (multi0-A -| T) node[right]{$B_0$};
\draw (multi0-B) -- (multi0-B -| T) node[right]{$B_1$};
%\foreach \s in {A,B} 
%\draw (multi1-\s) -- (multi1-\s -|  T) node[right]{$\s$};
\coordinate (A0) at (10.5,5);
\coordinate (A1) at (7.5,5);
\coordinate (A2) at (4.5,5);
\coordinate (A3) at (1.5,5);

\coordinate (N0) at (10.5,4);
\coordinate (N1) at (7.5,4);
\coordinate (N2) at (4.5,4);
\coordinate (N3) at (1.5,4);
\foreach \a in {0,1,2,3}
\draw (N\a) -- (A\a) node[above]{$A_\a$};
\foreach \m in {0,1,2,3} {
	\foreach \i in {00,01,10,11} {
		\draw (multi\m-\i) -- ++(90:0.5) coordinate (multi\m-\i-raised);
	}	
}
\draw (multi0-00-raised) -- (N0);
\draw (multi1-00-raised) -- (N1);
\draw (multi0-01-raised) -- (N1);
\draw (multi2-00-raised) -- (N2);
\draw (multi1-01-raised) -- (N2);
\draw (multi0-10-raised) -- (N2);
\draw (multi3-00-raised) -- (N3);
\draw (multi2-01-raised) -- (N3);
\draw (multi1-10-raised) -- (N3);
\draw (multi0-11-raised) -- (N3);

\draw (multi3-11-raised) -- (N3);
\draw (multi3-10-raised) -- (N3);
\draw (multi3-01-raised) -- (N3);
\draw (multi2-11-raised) -- (N3);
\draw (multi2-10-raised) -- (N3);
\draw (multi1-11-raised) -- (N3);

% output
\draw (multi3-output) -- ++(270:0.5) node[below]{$S_3$};
\draw (multi2-output) -- ++(270:0.5) node[below]{$S_2$};
\draw (multi1-output) -- ++(270:0.5) node[below]{$S_1$};
\draw (multi0-output) -- ++(270:0.5) node[below]{$S_0$};

\end{tikzpicture}

\end{document}